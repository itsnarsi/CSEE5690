
%% bare_conf.tex
%% V1.3
%% 2007/01/11
%% by Michael Shell
%% See:
%% http://www.michaelshell.org/
%% for current contact information.
%%
%% This is a skeleton file demonstrating the use of IEEEtran.cls
%% (requires IEEEtran.cls version 1.7 or later) with an IEEE conference paper.
%%
%% Support sites:
%% http://www.michaelshell.org/tex/ieeetran/
%% http://www.ctan.org/tex-archive/macros/latex/contrib/IEEEtran/
%% and
%% http://www.ieee.org/

%%*************************************************************************
%% Legal Notice:
%% This code is offered as-is without any warranty either expressed or
%% implied; without even the implied warranty of MERCHANTABILITY or
%% FITNESS FOR A PARTICULAR PURPOSE! 
%% User assumes all risk.
%% In no event shall IEEE or any contributor to this code be liable for
%% any damages or losses, including, but not limited to, incidental,
%% consequential, or any other damages, resulting from the use or misuse
%% of any information contained here.
%%
%% All comments are the opinions of their respective authors and are not
%% necessarily endorsed by the IEEE.
%%
%% This work is distributed under the LaTeX Project Public License (LPPL)
%% ( http://www.latex-project.org/ ) version 1.3, and may be freely used,
%% distributed and modified. A copy of the LPPL, version 1.3, is included
%% in the base LaTeX documentation of all distributions of LaTeX released
%% 2003/12/01 or later.
%% Retain all contribution notices and credits.
%% ** Modified files should be clearly indicated as such, including  **
%% ** renaming them and changing author support contact information. **
%%
%% File list of work: IEEEtran.cls, IEEEtran_HOWTO.pdf, bare_adv.tex,
%%                    bare_conf.tex, bare_jrnl.tex, bare_jrnl_compsoc.tex
%%*************************************************************************


\documentclass[conference,onecolumn]{IEEEtran}
\usepackage{blindtext, graphicx}


% *** GRAPHICS RELATED PACKAGES ***
%
\ifCLASSINFOpdf
  % \usepackage[pdftex]{graphicx}
  % declare the path(s) where your graphic files are
  % \graphicspath{{../pdf/}{../jpeg/}}
  % and their extensions so you won't have to specify these with
  % every instance of \includegraphics
  % \DeclareGraphicsExtensions{.pdf,.jpeg,.png}
\else
  % or other class option (dvipsone, dvipdf, if not using dvips). graphicx
  % will default to the driver specified in the system graphics.cfg if no
  % driver is specified.
  % \usepackage[dvips]{graphicx}
  % declare the path(s) where your graphic files are
  % \graphicspath{{../eps/}}
  % and their extensions so you won't have to specify these with
  % every instance of \includegraphics
  % \DeclareGraphicsExtensions{.eps}
\fi

\hyphenation{op-tical net-works semi-conduc-tor}


\begin{document}

\title{Forecasting CPU usage of Virtual Machines}
\author{\IEEEauthorblockN{Sai Narsi Reddy Donthi Reddy}
\IEEEauthorblockA{School of Electrical and\\Computer Engineering\\
University of Missouri -  Kansas City\\
Email: sdhy7@mail.umkc.edu}}

\maketitle

\IEEEpeerreviewmaketitle



\section{Introduction}
In this paper we forecast CPU usage by five different virtual machines by using different such as movine average (MA), weighted moving average (WMA), exponential smootheing(ES), and exponential smoothening with trend (EST). Along with these basic methods, we also test modified EST with weighted moving average for updating $F_t$ and peicewise regression models.

To test the performance of the forecasting methods we used tracking signal ratio (TS). Traking signal is the ratio of cumulative sum of forecasting signals to the mean absolute deviation. Which indicates the presents of bias in the results produced by the forecast model.

\begin{equation}
TS = \frac{\sum A_t - F_t}{MAD}
\end{equation}

Where $A_t$ is the actual value, $F_t$ is predicted value and MAD is the mean absolute deviation.

\begin{equation}
MAD = \frac{\sum |A_t - F_t|}{n}
\end{equation}

For a better forcasting model, TS should be not grater than 4 and not less than -4.
\begin{equation}
-4 < TS < +4 
\end{equation}

If traking singal value is out the limits, then the forecating model should be re-evaluated. If the TS above 4, most of our predictions as above actual values and vice versa.


\section{Forecasting Techniques}


\subsection{Moving Average (MA)}

\subsection{Weighted Moving Average (WMA)}

\subsection{Exponential Smoothening (ES)}

\subsection{Exponential Smoothening with Trend (EST)}

\subsection{Windowed Exponential Smoothening with Trend (WEST)}

\subsection{Peicewise regression (PR)}

\section{Results}

1) Show the frequency data for all five VM's

2) For each forecasting method talk about the results

3) Finally select the most varying VM's and show all the different methods performance.

\section{Conclusion}

Talk about what we just wrote in this paper.

% Can use something like this to put references on a page
% by themselves when using endfloat and the captionsoff option.
\ifCLASSOPTIONcaptionsoff
  \newpage
\fi



\begin{thebibliography}{1}

\bibitem{IEEEhowto:kopka}
H.~Kopka and P.~W. Daly, \emph{A Guide to \LaTeX}, 3rd~ed.\hskip 1em plus
  0.5em minus 0.4em\relax Harlow, England: Addison-Wesley, 1999.

\end{thebibliography}



\begin{IEEEbiography}[{\includegraphics[width=1in,height=1.25in,clip,keepaspectratio]{picture}}]{John Doe}
\blindtext
\end{IEEEbiography}





\end{document}


